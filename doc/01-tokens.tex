%
% This file is automatically generated from 01-tokens.json using gendoc.py.
% Do not edit by hand.
%
\texttt{} & \token{STRING} & Constante de cadena. \\
\texttt{} & \token{NUM} & Constante num\'erica. \\
\texttt{} & \token{LOWERID} & Identificador empezado con min\'usculas. \\
\texttt{} & \token{UPPERID} & Identificador empezado con may\'usculas. \\
\texttt{program} & \token{PROGRAM} & Para declarar la rutina principal. \\
\texttt{interactive} & \token{INTERACTIVE} & Para declarar una rutina principal interactiva con interactive program. \\
\texttt{procedure} & \token{PROCEDURE} & Para declarar procedimientos. \\
\texttt{function} & \token{FUNCTION} & Para declarar funciones. \\
\texttt{return} & \token{RETURN} & Para devolver valores de funciones y de la rutina principal. \\
\texttt{if} & \token{IF} & Para la alternativa condicional. \\
\texttt{then} & \token{THEN} & Palabra clave opcional para la rama 'then'. \\
\texttt{elseif} & \token{ELSEIF} & Para las ramas 'elseif'. \\
\texttt{else} & \token{ELSE} & Para la rama 'else'. \\
\texttt{choose} & \token{CHOOSE} & Para la expresi\'on alternativa condicional. \\
\texttt{when} & \token{WHEN} & Para las condiciones del 'choose'. \\
\texttt{otherwise} & \token{OTHERWISE} & Para la \'ultima rama del 'choose' y el 'matching..select'. \\
\texttt{repeat} & \token{REPEAT} & Repetici\'on simple. \\
\texttt{foreach} & \token{FOREACH} & Repetici\'on indexada. \\
\texttt{in} & \token{IN} & Para declarar el rango de la repetici\'on indexada. \\
\texttt{while} & \token{WHILE} & Repetici\'on condicional. \\
\texttt{switch} & \token{SWITCH} & Para hacer pattern matching. \\
\texttt{to} & \token{TO} & Palabra clave opcional despu\'es del sujeto sobre el que se hace pattern matching. \\
\texttt{matching} & \token{MATCHING} & Para la expresi\'on 'matching..select'. \\
\texttt{select} & \token{SELECT} & Para la expresi\'on 'matching..select'. \\
\texttt{on} & \token{ON} & Para la expresi\'on 'matching..select'. \\
\texttt{let} & \token{LET} & Palabra clave opcional para la asignaci\'on. Es obligatoria en el caso de las asignaciones de tuplas. \\
\texttt{not} & \token{NOT} & Negaci\'on l\'ogica. \\
\texttt{mod} & \token{MOD} & Resto de la divisi\'on entera. \\
\texttt{div} & \token{DIV} & Cociente de la divisi\'on entera. \\
\texttt{type} & \token{TYPE} & Para la declaraci\'on de nuevos tipos. \\
\texttt{is} & \token{IS} & Palabra clave para acompa\~nar la declaraci\'on de un nuevo tipo. \\
\texttt{record} & \token{RECORD} & Para tipos registro. \\
\texttt{variant} & \token{VARIANT} & Para tipos variantes. \\
\texttt{case} & \token{CASE} & Para las alternativas de tipos variantes. \\
\texttt{field} & \token{FIELD} & Para los campos de tipos registro. \\
\texttt{\_} & \token{UNDERSCORE} & Para marcar el caso default en un switch/match. \\
\texttt{TIMEOUT} & \token{TIMEOUT} & Para la rama timeout en un interactive program. \\
\texttt{\{} & \token{LBRACE} & Llave izquierda. \\
\texttt{\}} & \token{RBRACE} & Llave derecha. \\
\texttt{(} & \token{LPAREN} & Par\'entesis izquierdo. \\
\texttt{)} & \token{RPAREN} & Par\'entesis derecho. \\
\texttt{[} & \token{LBRACK} & Corchete izquierdo (para listas y rangos). \\
\texttt{]} & \token{RBRACK} & Corchete derecho. \\
\texttt{,} & \token{COMMA} & Coma. \\
\texttt{;} & \token{SEMICOLON} & Punto y coma (separador de instrucciones opcional). \\
\texttt{...} & \token{ELLIPSIS} & Fragmento de programa intencionalmente incompleto. \\
\texttt{..} & \token{RANGE} & Para rangos. \\
\texttt{:=} & \token{ASSIGN} & Asignaci\'on. \\
\texttt{\&\&} & \token{AND} & Conjunci\'on. \\
\texttt{||} & \token{OR} & Disyunci\'on. \\
\texttt{<-} & \token{GETS} & Para inicializaci\'on de campos. \\
\texttt{|} & \token{PIPE} & Para actualizaci\'on de campos. \\
\texttt{->} & \token{ARROW} & Usado en las ramas de un 'switch'. \\
\texttt{==} & \token{EQ} & Comparaci\'on por igualdad. \\
\texttt{/=} & \token{NE} & Comparaci\'on por desigualdad. \\
\texttt{<=} & \token{LE} & Comparaci\'on por menor o igual. \\
\texttt{>=} & \token{GE} & Comparaci\'on por mayor o igual. \\
\texttt{<} & \token{LT} & Comparaci\'on por menor estricto. \\
\texttt{>} & \token{LT} & Comparaci\'on por mayor estricto. \\
\texttt{++} & \token{CONCAT} & Operador de concatenaci\'on de listas. \\
\texttt{+} & \token{PLUS} & Suma de n\'umeros. \\
\texttt{-} & \token{MINUS} & Resta de n\'umeros y menos unario. \\
\texttt{*} & \token{TIMES} & Producto de n\'umeros. \\
\texttt{\^} & \token{POW} & Potencia. \\
%
% This file is automatically generated from 01-tokens.json using gendoc.py.
% Do not edit by hand.
%
