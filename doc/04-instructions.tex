%
% This file is automatically generated from 04-instructions.json using gendoc.py.
% Do not edit by hand.
%
\begin{itemize}
\item \instruction{PushInteger}(number : \type{INT})\\Mete la constante num\'erica dada en la pila.
\item \instruction{PushString}(string : \type{STRING})\\Mete la constante de cadena dada en la pila.
\item \instruction{PushVariable}(variableName : \type{ID})\\Mete el valor actual de la variable local variableName en la pila. Si variableName no tiene asociado un valor, lanza una excepci\'on GbsRuntimeError.
\item \instruction{SetVariable}(variableName : \type{ID})\\Saca el valor del tope de la pila y modifica la variable local variableName para que tome dicho valor. Si la variable no existe en el diccionario de nombres locales, se crea una entrada. Si la variable ya exist\'ia, se verifica que el tipo del valor que ten\'ia y el tipo del valor actual sean compatibles.
\item \instruction{UnsetVariable}(variableName : \type{ID})\\Elimina la variable local variableName del diccionario de nombres locales. Si la variable no existe en el diccionario de nombres locales, esta instrucci\'on no causa ning\'un otro efecto m\'as que incrementar el instruction pointer.
\item \instruction{Label}(label : \type{LABEL})\\Pseudo-instrucci\'on para definir una etiqueta, es decir un punto en el c\'odigo que puede ser el destino de un salto o una invocaci\'on. No puede haber etiquetas repetidas en todo el c\'odigo.
\item \instruction{Jump}(targetLabel : \type{LABEL})\\Salta incondicionalmente a la etiqueta indicada (debe existir).
\item \instruction{JumpIfFalse}(targetLabel : \type{LABEL})\\Si el tope de la pila es una estructura cuyo constructor es False, salta a la etiqueta indicada, que debe existir. Esta instrucci\'on saca el elemento del tope de la pila. La pila no puede estar vac\'ia.
\item \instruction{JumpIfStructure}(constructorName : \type{ID}, targetLabel : \type{LABEL})\\Si el tope de la pila es una estructura cuyo constructor es el indicado por 'constructorName', salta a la etiqueta indicada, que debe existir. No saca el elemento del tope de la pila. La pila no puede estar vac\'ia.
\item \instruction{JumpIfTuple}(size : \type{INT}, targetLabel : \type{LABEL})\\Si el tope de la pila es una tupla con la cantidad de componentes indicada por 'size', salta a la etiqueta indicada, que debe existir. No saca el elemento del tope de la pila. La pila no puede estar vac\'ia.
\item \instruction{Call}(targetLabel : \type{LABEL}, nargs : \type{INT})\\Hace una invocaci\'on a una subrutina. M\'as precisamente, mete un nuevo stack frame en la pila de llamadas, con el instruction pointer apuntando a la posici\'on del c\'odigo designada por la etiqueta 'targetLabel'. Saca 'nargs' valores del tope de la pila del stack frame llamador y los apila en la pila del nuevo stack frame. Notar que al desapilarlos y reapilarlos el orden se invierte, de tal manera que el primer par\'ametro queda en el tope de la pila.
\item \instruction{Return}()\\Retorna de una invocaci\'on a una subrutina. M\'as precisamente, saca el stack frame del tope de la pila de llamadas. Si la pila de llamadas queda vac\'ia, es el return del programa principal y el programa finaliza. Cuando se ejecuta la instrucci\'on Return, debe haber 0 o 1 valores en la pila. En caso de que haya un valor, se saca de la pila del stack frame actual y se apila en el stack frame del llamador.
\item \instruction{MakeTuple}(size : \type{INT})\\Crea una tupla del tama\~no indicado. Los elementos se sacan de la pila (el \'ultimo elemento de la tupla debe encontrarse en el tope de la pila).
\item \instruction{MakeList}(size : \type{INT})\\Crea una lista del tama\~no indicado. Los elementos se sacan de la pila (el \'ultimo elemento de la lista debe encontrarse en el tope de la pila). Los elementos de la lista deben tener tipos compatibles; de lo contrario se lanza una excepci\'on GbsRuntimeError.
\item \instruction{MakeStructure}(typeName : \type{ID}, constructorName : \type{ID}, fieldNames : [\type{ID}])\\Crea una estructura usando el constructor indicado ('constructorName') del tipo indicado ('typeName'), con los campos indicados por la lista de nombres 'fieldNames'. Los valores de cada campo se sacan de la pila (el valor del \'ultimo campo de la lista debe encontrarse en el tope de la pila).
\item \instruction{UpdateStructure}(typeName : \type{ID}, constructorName : \type{ID}, fieldNames : [\type{ID}])\\Actualiza una estructura usando el constructor indicado ('constructorName') del tipo indicado ('typeName'), con los campos indicados por la lista de nombres 'fieldNames'. Los valores de cada campo se sacan de la pila (el valor del \'ultimo campo de la lista debe encontrarse en el tope de la pila). A continuaci\'on se saca de la pila el valor, que debe ser una estructura para actualizar. Si el valor no es una estructura, o si es una estructura pero no est\'a construida con el constructor esperado, se lanza una excepci\'on GbsRuntimeError. Los tipos de los valores contenidos en los campos originales deben ser compatibles con los tipos de los nuevos valores de los respectivos campos; de lo contrario se lanza una excepci\'on GbsRuntimeError.
\item \instruction{ReadTupleComponent}(index : \type{INT})\\Mete en la pila la 'index'-\'esima componente de la tupla que se encuentra actualmente en el tope de la pila. Los \'indices empiezan desde 0. Debe haber un valor en el tope de la pila. No se saca dicho valor de la pila (queda como segundo elemento). Si el valor en el tope de la pila no es una tupla, se lanza una excepci\'on GbsRuntimeError. Si el \'indice est\'a fuera de rango, se lanza una excepci\'on GbsRuntimeError.
\item \instruction{ReadStructureField}(fieldName : \type{ID})\\Mete en la pila el valor del campo 'fieldName' de la estructura que se encuentra actualmente en el tope de la pila. Debe haber un valor en el tope de la pila. No se saca dicho valor de la pila (queda como segundo elemento). Si el valor en el tope de la pila no es una estructura, se lanza una excepci\'on GbsRuntimeError. Si el campo 'fieldName' no es uno de los campos presentes en la estructura, se lanza una excepci\'on GbsRuntimeError.
\item \instruction{ReadStructureFieldPop}(fieldName : \type{ID})\\Mete en la pila el valor del campo 'fieldName' de la estructura que se encuentra actualmente en el tope de la pila. Debe haber un valor en el tope de la pila. El valor original se saca de la pila. Si el valor en el tope de la pila no es una estructura, se lanza una excepci\'on GbsRuntimeError. Si el campo 'fieldName' no es uno de los campos presentes en la estructura, se lanza una excepci\'on GbsRuntimeError.
\item \instruction{PrimitiveCall}(primitiveName : \type{ID}, nargs : \type{INT})\\Invoca a una operaci\'on primitiva. Desapila 'nargs' argumentos de la pila (con el \'ultimo argumento en el tope). A continuaci\'on, invoca a la primitiva pas\'andole como par\'ametros el estado global, seguido de los 'nargs' argumentos desapilados. La funci\'on primitiva puede devolver un valor o 'null'. En caso de que devuelva un valor, se apila dicho resultado en la pila. Si devuelve null, prosigue con la ejecuci\'on.
\item \instruction{SaveState}()\\Crea una copia el estado global actual y lo mete en la pila de estados globales.
\item \instruction{RestoreState}()\\Saca un estado de la pila de estados globales.
\item \instruction{TypeCheck}(type : \type{TYPE})\\Verifica que el tipo del valor del tope de la pila sea compatible con el tipo indicado. La pila no debe estar vac\'ia. Esta operaci\'on no saca el elemento del tope de la pila.
\item \TODO{TODO}
\end{itemize}
%
% This file is automatically generated from 04-instructions.json using gendoc.py.
% Do not edit by hand.
%
