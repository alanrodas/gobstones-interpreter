%
% This file is automatically generated from 04-instructions.json using gendoc.py.
% Do not edit by hand.
%
\begin{itemize}
\item \instruction{PushConstant}(constant : \type{STRING} + \type{INT})\\Mete la constante dada en la pila.
\item \instruction{PushVariable}(variableName : \type{ID})\\Mete el valor actual de la variable local variableName en la pila. Si variableName no tiene asociado un valor, lanza una excepci\'on GbsRuntimeError.
\item \instruction{SetVariable}(variableName : \type{ID})\\Saca el valor del tope de la pila y modifica la variable local variableName para que tome dicho valor. Si la variable no existe en el diccionario de nombres locales, se crea una entrada.
\item \instruction{UnsetVariable}(variableName : \type{ID})\\Elimina la variable local variableName del diccionario de nombres locales. Si la variable no existe en el diccionario de nombres locales, esta instrucci\'on no causa ning\'un otro efecto m\'as que incrementar el instruction pointer.
\item \instruction{Label}(label : \type{LABEL})\\Pseudo-instrucci\'on para definir una etiqueta, es decir un punto en el c\'odigo que puede ser el destino de un salto o una invocaci\'on. No puede haber etiquetas repetidas en todo el c\'odigo.
\item \instruction{Jump}(targetLabel : \type{LABEL})\\Salta incondicionalmente a la etiqueta indicada (debe existir).
\item \instruction{JumpIfFalse}(targetLabel : \type{LABEL})\\Si el tope de la pila es False, salta a la etiqueta indicada, que debe existir. Esta instrucci\'on saca el elemento del tope de la pila. La pila no puede estar vac\'ia.
\item \instruction{JumpIfConstructor}(constructorName : \type{ID}, targetLabel : \type{LABEL})\\Si el tope de la pila es una instancia del constructor indicado por 'constructorName', salta a la etiqueta indicada, que debe existir. No saca el elemento del tope de la pila. La pila no puede estar vac\'ia.
\item \instruction{JumpIfTuple}(size : \type{INT}, targetLabel : \type{LABEL})\\Si el tope de la pila es una tupla con la cantidad de componentes indicada por 'size', salta a la etiqueta indicada, que debe existir. No saca el elemento del tope de la pila. La pila no puede estar vac\'ia.
\item \TODO{TODO}
\end{itemize}
%
% This file is automatically generated from 04-instructions.json using gendoc.py.
% Do not edit by hand.
%
