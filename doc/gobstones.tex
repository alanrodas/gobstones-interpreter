\documentclass{article}
\usepackage{geometry}
\usepackage{xcolor}
\usepackage{xspace}
\usepackage{amsmath}
\usepackage{tabularx}
\usepackage{longtable}
\usepackage[spanish]{babel}
\begin{document}
\colorlet{darkgreen}{green!60!black}
\newcommand{\chr}[1]{\texttt{'}\textcolor{blue}{\texttt{#1}}\texttt{'}}
\newcommand{\str}[1]{\texttt{"}\textcolor{blue}{\texttt{#1}}\texttt{"}}
\newcommand{\token}[1]{\textcolor{red}{\texttt{#1}}}
\newcommand{\nonterminal}[1]{\textcolor{blue}{{\it$\langle$#1$\rangle$}}}
\newcommand{\nonEmpty}[1]{#1$_{1}$}
\newcommand{\production}[2]{
  \noindent
  \begin{tabular}{lrl}
  #1 & $\xrightarrow{\hspace{.5cm}}$ & #2
  \end{tabular}\\
}
\newcommand{\EMPTY}{$\epsilon$}
\newcommand{\ALT}{
  \\ & $\mid$ &
}
\newcommand{\ALTA}{
  $\mid$
}
\newcommand{\TODO}[1]{\textcolor{red}{****#1****}}

\newcommand{\type}[1]{\textcolor{blue}{\texttt{#1}}}
\renewcommand{\ast}[1]{\textcolor{darkgreen}{\texttt{\underline{#1}}}}
\newcommand{\typedecl}[2]{\noindent
  \begin{tabularx}{\textwidth}{lrlr}
  #1 & $=$ & #2
  \end{tabularx}\\
}
\newcommand{\datadecl}[2]{\noindent
  \begin{tabularx}{\textwidth}{lrp{13cm}r}
  #1 & $::=$ & #2
  \end{tabularx}\\
}

\newcommand{\PUEDE}{{\bf PUEDE}\xspace}
\newcommand{\PUEDEN}{{\bf PUEDEN}\xspace}
\newcommand{\NOPUEDE}{{\bf NO PUEDE}\xspace}
\newcommand{\NOPUEDEN}{{\bf NO PUEDEN}\xspace}
\newcommand{\DEBE}{{\bf DEBE}\xspace}
\newcommand{\DEBEN}{{\bf DEBEN}\xspace}

\section{Gram\'atica de XGobstones}

\subsection{Sintaxis l\'exica}

\begin{itemize}
\item Se ignoran:
\chr{\textbackslash t} (\texttt{TAB}, chr 9),
\chr{\textbackslash n} (\texttt{LF}, chr 10),
\chr{\textbackslash r} (\texttt{CR}, chr 13),
\chr{\,} (\texttt{SPACE}, chr 32).
\item Se admiten comentarios comenzados por \str{//} (estilo C++), por \str{--} (estilo Haskell) y por \str{\#} (estilo shell) que se extienden hasta el fin de l\'inea (\texttt{LF}).
\item Se admiten comentarios delimitados por \str{/*} \str{*/} y por \str{\{-} \str{-\}} que pueden anidarse.
\item El tokenizador reconoce directivas {\em pragma} de la forma \str{/*@parte$_1$@parte$_2$@...@parte$_n$@*/}. La idea es que esto pueda ser un mecanismo extensible de directivas. Ver m\'as abajo las directivas soportadas.
\end{itemize}


\subsection{Lista de s\'imbolos terminales}

Todos los s\'imbolos terminales se acompa\~nan de su nombre \token{EN\_MAYUSCULAS}.\bigskip

\begin{itemize}

\item {\bf Constantes num\'ericas.}
  Son una secuencia de d\'igitos decimales que representan enteros positivos.
  No pueden tener ceros innecesarios a la izquierda.
  \begin{center}
    \token{NUM} ::= \texttt{0|[1-9][0-9]*}
  \end{center}

\item {\bf Identificadores.}
  Comienzan por un caracter alfab\'etico y
  est\'an seguidos por cero o m\'as caracteres,
  ya sea alfab\'eticos, num\'ericos, gui\'on bajo (\texttt{\_}) o comilla simple (\texttt{'}).
  Se aceptan caracteres Unicode arbitrarios. Decimos que un caracter $c$ es {\bf alfab\'etico}
  cuando tiene dos variantes, una min\'uscula y una may\'uscula, es decir cuando:
  \[
  \texttt{$c$.toUpperCase() != $c$.toLowerCase()}
  \]
  Los identificadores \token{LOWERID} comienzan por un caracter en min\'usculas
  (y sirven para identificar \'indices, par\'ametros, funciones, variables, campos).
  Los identificadores \token{UPPERID} comienzan por un caracter en may\'usculas
  (y sirven para identificador constructores, procedimientos, tipos).
  \begin{center}
    \token{LOWERID} ::= \texttt{[:lower:]([:alpha:]|[0-9]|\_|')*}
  \end{center}
  \begin{center}
    \token{UPPERID} ::= \texttt{[:upper:]([:alpha:]|[0-9]|\_|')*}
  \end{center}

\item {\bf Constantes de cadena.}
  Las cadenas est\'an delimitadas por comillas dobles (\chr{"}).
  Todos los caracteres desde la comilla que abre hasta la que cierra se
  toman literalmente salvo la contrabarra (\chr{\textbackslash}) que se
  interpreta como un caracter de escape.
  Los escapes admitidos son:
  \begin{itemize}
  \item \texttt{\textbackslash\textbackslash} -- contrabarra (\textbackslash)
  \item \texttt{\textbackslash"} -- comilla doble (")
  \item \texttt{\textbackslash{a}} -- \texttt{BEL}, chr 7
  \item \texttt{\textbackslash{b}} -- \texttt{BS}, chr 8
  \item \texttt{\textbackslash{f}} -- \texttt{FF}, chr 12
  \item \texttt{\textbackslash{n}} -- \texttt{LF}, chr 10
  \item \texttt{\textbackslash{r}} -- \texttt{CR}, chr 13
  \item \texttt{\textbackslash{t}} -- \texttt{TAB}, chr 9
  \item \texttt{\textbackslash{v}} -- \texttt{VT}, chr 11
  \end{itemize}
  \begin{center}
    \token{STRING} ::= \texttt{"(\textbackslash{a}|\textbackslash{b}|\textbackslash{f}|\textbackslash{n}|\textbackslash{r}|\textbackslash{t}|\textbackslash{v}|\textbackslash\textbackslash|\textbackslash"|[\^{}\textbackslash"])*"}
  \end{center}

\item {\bf Lista completa de tokens.}
  \begin{longtable}{llp{.6\textwidth}}
  % El siguiente tex se genera a partir de 01-tokens.json usando gendoc.py.
  %
% This file is automatically generated from 01-tokens.json using gendoc.py.
% Do not edit by hand.
%
\texttt{} & \token{STRING} & Constante de cadena. \\
\texttt{} & \token{NUM} & Constante num\'erica. \\
\texttt{} & \token{LOWERID} & Identificador empezado con min\'usculas. \\
\texttt{} & \token{UPPERID} & Identificador empezado con may\'usculas. \\
\texttt{program} & \token{PROGRAM} & Para declarar la rutina principal. \\
\texttt{interactive} & \token{INTERACTIVE} & Para declarar una rutina principal interactiva con interactive program. \\
\texttt{procedure} & \token{PROCEDURE} & Para declarar procedimientos. \\
\texttt{function} & \token{FUNCTION} & Para declarar funciones. \\
\texttt{return} & \token{RETURN} & Para devolver valores de funciones y de la rutina principal. \\
\texttt{if} & \token{IF} & Para la alternativa condicional. \\
\texttt{then} & \token{THEN} & Palabra clave opcional para la rama 'then'. \\
\texttt{else} & \token{ELSE} & Para la rama 'else'. \\
\texttt{repeat} & \token{REPEAT} & Repetici\'on simple. \\
\texttt{foreach} & \token{FOREACH} & Repetici\'on indexada. \\
\texttt{in} & \token{IN} & Para declarar el rango de la repetici\'on indexada. \\
\texttt{while} & \token{WHILE} & Repetici\'on condicional. \\
\texttt{switch} & \token{SWITCH} & Para hacer pattern matching. \\
\texttt{match} & \token{SWITCH} & Alternativa para 'switch'. \\
\texttt{to} & \token{TO} & Palabra clave opcional despu\'es del sujeto sobre el que se hace pattern matching. \\
\texttt{let} & \token{LET} & Palabra clave opcional para la asignaci\'on. Es obligatoria en el caso de las asignaciones de tuplas. \\
\texttt{not} & \token{NOT} & Negaci\'on l\'ogica. \\
\texttt{mod} & \token{MOD} & Resto de la divisi\'on entera. \\
\texttt{div} & \token{DIV} & Cociente de la divisi\'on entera. \\
\texttt{type} & \token{TYPE} & Para la declaraci\'on de nuevos tipos. \\
\texttt{is} & \token{IS} & Palabra clave para acompa\~nar la declaraci\'on de un nuevo tipo. \\
\texttt{record} & \token{RECORD} & Para tipos registro. \\
\texttt{variant} & \token{VARIANT} & Para tipos variantes. \\
\texttt{case} & \token{CASE} & Para las alternativas de tipos variantes. \\
\texttt{field} & \token{FIELD} & Para los campos de tipos registro. \\
\texttt{\_} & \token{UNDERSCORE} & Para marcar el caso default en un switch/match. \\
\texttt{TIMEOUT} & \token{TIMEOUT} & Para la rama timeout en un interactive program. \\
\texttt{\{} & \token{LBRACE} & Llave izquierda. \\
\texttt{\}} & \token{RBRACE} & Llave derecha. \\
\texttt{(} & \token{LPAREN} & Par\'entesis izquierdo. \\
\texttt{)} & \token{RPAREN} & Par\'entesis derecho. \\
\texttt{[} & \token{LBRACK} & Corchete izquierdo (para listas y rangos). \\
\texttt{]} & \token{RBRACK} & Corchete derecho. \\
\texttt{,} & \token{COMMA} & Coma. \\
\texttt{;} & \token{SEMICOLON} & Punto y coma (separador de instrucciones opcional). \\
\texttt{..} & \token{RANGE} & Para rangos. \\
\texttt{:=} & \token{ASSIGN} & Asignaci\'on. \\
\texttt{\&\&} & \token{AND} & Conjunci\'on. \\
\texttt{||} & \token{OR} & Disyunci\'on. \\
\texttt{<-} & \token{GETS} & Para inicializaci\'on de campos. \\
\texttt{|} & \token{PIPE} & Para actualizaci\'on de campos. \\
\texttt{->} & \token{ARROW} & Usado en las ramas de un 'switch'. \\
\texttt{==} & \token{EQ} & Comparaci\'on por igualdad. \\
\texttt{/=} & \token{NE} & Comparaci\'on por desigualdad. \\
\texttt{<=} & \token{LE} & Comparaci\'on por menor o igual. \\
\texttt{>=} & \token{GE} & Comparaci\'on por mayor o igual. \\
\texttt{<} & \token{LT} & Comparaci\'on por menor estricto. \\
\texttt{>} & \token{LT} & Comparaci\'on por mayor estricto. \\
\texttt{++} & \token{CONCAT} & Operador de concatenaci\'on de listas. \\
\texttt{+} & \token{PLUS} & Suma de n\'umeros. \\
\texttt{-} & \token{MINUS} & Resta de n\'umeros y menos unario. \\
\texttt{*} & \token{TIMES} & Producto de n\'umeros. \\
\texttt{\^} & \token{POW} & Potencia. \\
%
% This file is automatically generated from 01-tokens.json using gendoc.py.
% Do not edit by hand.
%

  \end{longtable}
\end{itemize}

\subsection{Pragmas}

El tokenizador implementa directivas pragma para
mantener registro de ``regiones''. Una regi\'on es una cadena de texto que
sirve para identificar o {\em taggear} un fragmento del programa.
Las regiones no tienen ning\'un significado
especial para el int\'erprete de Gobstones, pero todas las excepciones que
eleva el int\'erprete vienen acompa\~nadas de una posici\'on que incluye
el nombre de la regi\'on actual. Las regiones pueden anidarse.
Este comportamiento se implementa por medio de dos directivas.
\begin{itemize}
\item \str{/*@BEGIN\_REGION@\textit{nombre\_de\_la\_regi\'on}@*/}:
       mete el nombre de una regi\'on en la pila de regiones.
\item \str{/*@END\_REGION@*/}: saca la regi\'on del tope de la pila de regiones.
\end{itemize}
Por ejemplo ante el siguiente programa:
\begin{verbatim}
/*@BEGIN_REGION@A@*/
procedure P() {
  /*@BEGIN_REGION@B@*/
  x := f(
  /*@END_REGION@*/
}
/*@END_REGION@*/
\end{verbatim}
El parser idealmente deber\'ia reportar que hay un error de sintaxis en la regi\'on \texttt{B}.

\subsection{Gram\'atica}
Los s\'imbolos no terminales se describen con su nombre \nonterminal{enCursiva}.
Las producciones se escriben siguiendo las convenciones usuales de EBNF.
La gram\'atica es liberal en algunos sentidos. En particular:
\begin{itemize}
\item El \texttt{return}
se considera un {\em statement} que puede aparecer en la posici\'on en la que
podr\'ia aparecer cualquier otra instrucci\'on. La restricci\'on de que
el \texttt{return} \'unicamente aparezca como \'ultima instrucci\'on del bloque, y
\'unicamente al final de las declaraciones de funciones y del programa principal,
es una restricci\'on que se posterga para la etapa de chequeo sem\'antico
({\em lint}).
\item Se admite el gui\'on bajo (\chr{\_}) como un posible patr\'on en el lado
izquierdo de las ramas de un \texttt{switch} y en el lado izquierdo de las ramas de un.
\texttt{interactive program}.
En una lista de ramas deber\'ia haber un \'unico gui\'on bajo, y deber\'ia ser el \'ultimo
patr\'on de la lista pero esto, de nuevo, se relega a la etapa de lint.
\item Los patrones del interactive program (\texttt{TIMEOUT}, \texttt{K\_ENTER}, etc.) se
admiten como patrones en cualquier switch.
\end{itemize}

Las convenciones de asociatividad y precedencia de operadores no se reflejan
en las producciones, sino en la tabla de precedencia.
(Para expresar esto en la gram\'atica ser\'ia necesario estratificar las expresiones
en t\'erminos, factores, \'atomos, etc., tal como se hace en la gram\'atica oficial).
\bigskip

% El siguiente tex se genera a partir de 02-grammar.json usando gendoc.py.
%
% This file is automatically generated from 02-grammar.json using gendoc.py.
% Do not edit by hand.
%
\production{\nonterminal{start}}{
(\nonterminal{definition})*
}
\production{\nonterminal{definition}}{
\nonterminal{defProgram}
\ALT
\nonterminal{defInteractiveProgram}
\ALT
\nonterminal{defProcedure}
\ALT
\nonterminal{defFunction}
\ALT
\nonterminal{defType}
}
\production{\nonterminal{defProgram}}{
\token{PROGRAM} \nonterminal{stmtBlock}
}
\production{\nonterminal{defInteractiveProgram}}{
\token{INTERACTIVE} \token{PROGRAM} (\nonterminal{stmtSwitchBranch})* \token{RBRACE}
}
\production{\nonterminal{defProcedure}}{
\token{PROCEDURE} \token{LPAREN} \nonterminal{loweridSeq} \token{RPAREN} \nonterminal{stmtBlock}
}
\production{\nonterminal{defFunction}}{
\token{FUNCTION} \token{LPAREN} \nonterminal{loweridSeq} \token{RPAREN} \nonterminal{stmtBlock}
}
\production{\nonterminal{defType}}{
\token{FUNCTION} \token{LPAREN} \nonterminal{loweridSeq} \token{RPAREN} \nonterminal{stmtBlock}
\ALT
\token{TYPE} \token{UPPERID} \token{IS} \token{RECORD} \token{LBRACE} (\nonterminal{fieldDeclaration})* \token{RBRACE}
\ALT
\token{TYPE} \token{UPPERID} \token{IS} \token{VARIANT} \token{LBRACE} (\nonterminal{constructorDeclaration})* \token{RBRACE}
}
\production{\nonterminal{constructorDeclaration}}{
\token{CASE} \token{UPPERID} \token{LBRACE} (\nonterminal{fieldDeclaration})* \token{RBRACE}
}
\production{\nonterminal{fieldDeclaration}}{
\token{FIELD} \token{LOWERID}
}
\production{\nonterminal{loweridSeq}}{
\EMPTY
\ALT
\nonterminal{\nonEmpty{loweridSeq}}
}
\production{\nonterminal{\nonEmpty{loweridSeq}}}{
\token{LOWERID} (\token{COMMA} \nonterminal{\nonEmpty{loweridSeq}})?
}
\production{\nonterminal{statement}}{
\nonterminal{stmtBlock}
\ALT
\nonterminal{stmtReturn}
\ALT
\nonterminal{stmtIf}
\ALT
\nonterminal{stmtRepeat}
\ALT
\nonterminal{stmtForeach}
\ALT
\nonterminal{stmtWhile}
\ALT
\nonterminal{stmtSwitch}
\ALT
\nonterminal{stmtLet}
\ALT
\nonterminal{stmtVariableAssignment}
\ALT
\nonterminal{stmtProcedureCall}
}
\production{\nonterminal{stmtBlock}}{
\token{LBRACE} (\nonterminal{statement} (\token{SEMICOLON})?)* \token{RBRACE}
}
\production{\nonterminal{stmtReturn}}{
\token{RETURN} \token{LPAREN} \nonterminal{expressionSeq} \token{RPAREN}
}
\production{\nonterminal{stmtIf}}{
\token{IF} \nonterminal{stmtIfBranch} (\token{ELSEIF} \nonterminal{stmtIfBranch})* (\token{ELSE} \nonterminal{stmtBlock})?
}
\production{\nonterminal{stmtIfBranch}}{
\token{LPAREN} \nonterminal{expression} \token{RPAREN} (\token{THEN})? \nonterminal{stmtBlock}
}
\production{\nonterminal{stmtRepeat}}{
\token{REPEAT} \token{LPAREN} \nonterminal{expression} \token{RPAREN} \nonterminal{stmtBlock}
}
\production{\nonterminal{stmtForeach}}{
\token{FOREACH} \nonterminal{pattern} \token{IN} \nonterminal{expression} \nonterminal{stmtBlock}
}
\production{\nonterminal{stmtWhile}}{
\token{WHILE} \token{LPAREN} \nonterminal{expression} \token{RPAREN} \nonterminal{stmtBlock}
}
\production{\nonterminal{stmtSwitch}}{
\token{SWITCH} \token{LPAREN} \nonterminal{expression} \token{RPAREN} (\token{TO})? \token{LBRACE} (\nonterminal{stmtSwitchBranch})* \token{RBRACE}
}
\production{\nonterminal{stmtSwitchBranch}}{
\nonterminal{pattern} \token{ARROW} \nonterminal{stmtBlock}
}
\production{\nonterminal{stmtLet}}{
\token{LET} \nonterminal{stmtVariableAssignment}
\ALT
\token{LET} \nonterminal{stmtTupleAssignment}
}
\production{\nonterminal{stmtVariableAssignment}}{
\token{LOWERID} \token{ASSIGN} \nonterminal{expression}
}
\production{\nonterminal{stmtTupleAssignment}}{
\token{LPAREN} \token{RPAREN} \token{ASSIGN} \nonterminal{expression}
\ALT
\token{LPAREN} \token{LOWERID} \token{COMMA} \nonterminal{\nonEmpty{loweridSeq}} \token{RPAREN} \token{ASSIGN} \nonterminal{expression}
}
\production{\nonterminal{stmtProcedureCall}}{
\token{UPPERID} \token{LPAREN} \nonterminal{expressionSeq} \token{RPAREN}
}
\production{\nonterminal{pattern}}{
\nonterminal{patternWildcard}
\ALT
\nonterminal{patternVariable}
\ALT
\nonterminal{patternNumber}
\ALT
\nonterminal{patternStructure}
\ALT
\nonterminal{patternTuple}
\ALT
\nonterminal{patternTimeout}
}
\production{\nonterminal{patternWildcard}}{
\token{UNDERSCORE}
}
\production{\nonterminal{patternVariable}}{
\token{LOWERID}
}
\production{\nonterminal{patternNumber}}{
\token{NUM}
\ALT
\token{MINUS} \token{NUM}
}
\production{\nonterminal{patternStructure}}{
\token{UPPERID} (\token{LPAREN} \nonterminal{loweridSeq} \token{RPAREN})?
}
\production{\nonterminal{patternTuple}}{
\token{LPAREN} \token{RPAREN}
\ALT
\token{LPAREN} \token{LOWERID} \token{COMMA} \nonterminal{\nonEmpty{loweridSeq}} \token{RPAREN}
}
\production{\nonterminal{patternTimeout}}{
\token{TIMEOUT} \token{LPAREN} \token{NUM} \token{RPAREN}
}
\production{\nonterminal{expression}}{
\nonterminal{exprAtom}
\ALT
\nonterminal{expression} \nonterminal{infixOperator} \nonterminal{expression}
\ALT
\nonterminal{prefixOperator} \nonterminal{expression}
\ALT
\token{LPAREN} \nonterminal{expression} \token{RPAREN}
}
\production{\nonterminal{exprAtom}}{
\nonterminal{exprVariable}
\ALT
\nonterminal{exprFunctionCall}
\ALT
\nonterminal{exprConstantNumber}
\ALT
\nonterminal{exprConstantString}
\ALT
\nonterminal{exprChoose}
\ALT
\nonterminal{exprList}
\ALT
\nonterminal{exprRange}
\ALT
\nonterminal{exprTuple}
\ALT
\nonterminal{exprStructure}
\ALT
\nonterminal{exprStructureUpdate}
}
\production{\nonterminal{exprVariable}}{
\token{LOWERID}
}
\production{\nonterminal{exprFunctionCall}}{
\token{LOWERID} \token{LPAREN} \nonterminal{expressionSeq} \token{RPAREN}
}
\production{\nonterminal{exprConstantNumber}}{
\token{NUM}
}
\production{\nonterminal{exprConstantString}}{
\token{STRING}
}
\production{\nonterminal{exprChoose}}{
\token{CHOOSE} (\nonterminal{expression} \token{WHEN} \token{LPAREN} \nonterminal{expression} \token{RPAREN})* \nonterminal{expression} \token{OTHERWISE}
}
\production{\nonterminal{exprList}}{
\token{LBRACK} \nonterminal{expressionSeq} \token{RBRACK}
}
\production{\nonterminal{exprRange}}{
\token{LBRACK} \nonterminal{expression} \token{RANGE} \nonterminal{expression} \token{RBRACK}
\ALT
\token{LBRACK} \nonterminal{expression} \token{COMMA} \nonterminal{expression} \token{RANGE} \nonterminal{expression} \token{RBRACK}
}
\production{\nonterminal{exprTuple}}{
\token{LPAREN} \token{RPAREN}
\ALT
\token{LPAREN} \nonterminal{expression} \token{COMMA} \nonterminal{\nonEmpty{expressionSeq}} \token{RPAREN}
}
\production{\nonterminal{exprStructure}}{
\token{UPPERID} (\token{LPAREN} \nonterminal{fieldBindingSeq} \token{RPAREN})?
}
\production{\nonterminal{exprStructureUpdate}}{
\token{UPPERID} \token{LPAREN} \nonterminal{expression} \token{PIPE} \nonterminal{fieldBindingSeq} \token{RPAREN}
}
\production{\nonterminal{fieldBinding}}{
\token{LOWERID} \token{GETS} \nonterminal{expression}
}
\production{\nonterminal{fieldBindingSeq}}{
\EMPTY
\ALT
\nonterminal{\nonEmpty{fieldBindingSeq}}
}
\production{\nonterminal{\nonEmpty{fieldBindingSeq}}}{
\nonterminal{fieldBinding} (\token{COMMA} \nonterminal{\nonEmpty{fieldBindingSeq}})?
}
\production{\nonterminal{infixOperator}}{
\token{OR}
\ALT
\token{AND}
\ALT
\token{EQ}
\ALT
\token{NE}
\ALT
\token{LE}
\ALT
\token{GE}
\ALT
\token{LT}
\ALT
\token{GT}
\ALT
\token{CONCAT}
\ALT
\token{PLUS}
\ALT
\token{MINUS}
\ALT
\token{TIMES}
\ALT
\token{DIV}
\ALT
\token{MOD}
\ALT
\token{POW}
}
\production{\nonterminal{prefixOperator}}{
\token{MINUS}
\ALT
\token{NOT}
}
\production{\nonterminal{expressionSeq}}{
\EMPTY
\ALT
\nonterminal{\nonEmpty{expressionSeq}}
}
\production{\nonterminal{\nonEmpty{expressionSeq}}}{
\nonterminal{expression} (\token{COMMA} \nonterminal{\nonEmpty{expressionSeq}})?
}
%
% This file is automatically generated from 02-grammar.json using gendoc.py.
% Do not edit by hand.
%


\subsubsection{Precedencia de operadores}

Los operadores se organizan seg\'un la siguiente tabla.
Cada fila corresponde a un nivel de precedencia, ordenados de menor a mayor precedencia.
Las {\em fixities} posibles son:
operador binario asociativo a derecha ({\bf InfixR}),
operador binario asociativo a izquierda ({\bf InfixL}),
operador binario no asociativo ({\bf Infix}),
operador unario prefijo ({\bf Prefix}).

\begin{itemize}
\item {\bf InfixR:} \token{OR} (\texttt{||})
\item {\bf InfixR:} \token{AND} (\texttt{\&\&})
\item {\bf Prefix:} \token{NOT} (\texttt{not})
\item {\bf Infix:}
  \token{EQ} (\texttt{==})
  \token{NE} (\texttt{/=})
  \token{LE} (\texttt{<=})
  \token{GE} (\texttt{>=})
  \token{LT} (\texttt{<})
  \token{GT} (\texttt{>})
\item {\bf InfixL:}
  \token{CONCAT} (\texttt{++})
\item {\bf InfixL:}
  \token{PLUS} (\texttt{+})
  \token{MINUS} (\texttt{-})
\item {\bf InfixL:}
  \token{TIMES} (\texttt{*})
\item {\bf InfixL:}
  \token{DIV} (\texttt{div})
  \token{MOD} (\texttt{mod})
\item {\bf InfixR:}
  \token{POW} (\texttt{\^})
\item {\bf Prefix:}
  \token{MINUS} (\texttt{-}, menos unario)
\end{itemize}

\subsection{\'Arbol de sintaxis abstracta}

El resultado de analizar sint\'acticamente un programa es un objeto.
El objeto representa un \'arbol sint\'actico y est\'a construido
recursivamente de la siguiente manera:
\begin{itemize}
\item Las hojas del \'arbol son instancias de la clase \texttt{Token}
      cuyo atributo 
      \texttt{tag} representa el tipo de token (por ejemplo,
      \texttt{T\_UPPERID})
      y cuyo atributo \texttt{value} representa el valor le\'ido
      (por ejemplo, ``\texttt{PonerN}'').
\item Los nodos internos del \'arbol son instancias de la clase
      \texttt{ASTNode} o, m\'as precisamente, de alguna de sus subclases.
      Un nodo interno tiene dos atributos:
      \begin{itemize}
      \item \texttt{tag}: indica el ``tipo'' de nodo del que se trata.
            Por ejemplo
            una instancia de la clase \texttt{ASTStmtIf}
            representa un comando \texttt{if-then-else},
            y el tag asociado es el s\'imbolo \texttt{N\_StmtIf}.
      \item \texttt{children}: es la lista de hijos del nodo.
            Por ejemplo,
            las instancias de la clase \texttt{ASTStmtIf}
            tienen exactamente tres hijos:
            el primero es una expresi\'on que representa la condici\'on
            del if,
            el segundo es un comando que representa la rama \texttt{then},
            y el tercero puede ser \texttt{null} (si la rama \texttt{else})
            se encuentra ausente), o un comando que representa la rama \texttt{else}.
      \end{itemize}
\item Como se mencion\'o arriba, en algunos (pocos) casos
      las hojas del \'arbol tambi\'en pueden ser \texttt{null},
      que se usa para indicar la ausencia de algunas
      componentes opcionales del \'arbol (tales como la rama \texttt{else}
      de un \texttt{if}).
\item Los hijos eventualmente tambi\'en pueden ser listas de ASTs.
\end{itemize}

Abajo se especifica la forma que tiene un \'arbol usando una sintaxis
similar a la de la declaraci\'on de un tipo de datos inductivo en
Haskell.
Los tipos escritos \type{EN\_MAY\'USCULAS} representan el tipo de los tokens.
Los tipos escritos \type{EnMin\'usculas} representan categor\'ias
abstractas de ASTs
(por ejemplo, \type{Statement} es la categor\'ia abstracta de aquellos
AST que representan comandos).
Las palabras escritas \ast{EnVerde} representan categor\'ias concretas
(en la terminolog\'ia de Haskell, constructores; en el caso de JavaScript
corresponden a subclases de \texttt{ASTNode}).
Por ejemplo, \ast{StmtIf} es la categor\'ia concreta de los AST que
representan un comando if-then-else.
En JavaScript, su clase asociada se llama \texttt{ASTStmtIf},
y su tag asociado es el s\'imbolo \texttt{N\_StmtIf}.

% El siguiente tex se genera a partir de 03-ast.json usando gendoc.py.
%
% This file is automatically generated from 03-ast.json using gendoc.py.
% Do not edit by hand.
%
\typedecl{\type{ID}}{String}
\typedecl{\type{NUM}}{Integer}
\typedecl{\type{STRING}}{String}
\datadecl{\type{Main}}{\ast{Main}(definitions : [\type{Definition}])}
\datadecl{\type{Definition}}{\ast{DefProgram}(body : \type{Statement})
\ALT
\ast{DefInteractiveProgram}(branches : [\type{SwitchBranch}])
\ALT
\ast{DefProcedure}(name : \type{ID}, parameters : [\type{ID}], body : \type{Statement})
\ALT
\ast{DefFunction}(name : \type{ID}, parameters : [\type{ID}], body : \type{Statement})
\ALT
\ast{DefType}(typeName : \type{ID}, constructorDeclarations : [\type{ConstructorDeclaration}])}
\datadecl{\type{Statement}}{\ast{StmtBlock}(statements : [\type{Statement}])
\ALT
\ast{StmtReturn}(result : \type{Expression})
\ALT
\ast{StmtIf}(condition : \type{Expression}, thenBlock : \type{Statement}, elseBlock : (\type{Statement})?)
\ALT
\ast{StmtRepeat}(times : \type{Expression}, body : \type{Statement})
\ALT
\ast{StmtForeach}(pattern : \type{Pattern}, range : \type{Expression}, body : \type{Statement})
\ALT
\ast{StmtWhile}(condition : \type{Expression}, body : \type{Statement})
\ALT
\ast{StmtSwitch}(subject : \type{Expression}, branches : [\type{SwitchBranch}])
\ALT
\ast{StmtAssignVariable}(variable : \type{ID}, value : \type{Expression})
\ALT
\ast{StmtAssignTuple}(variables : [\type{ID}], value : \type{Expression})
\ALT
\ast{StmtProcedureCall}(procedureName : \type{ID}, args : [\type{Expression}])}
\datadecl{\type{SwitchBranch}}{\ast{SwitchBranch}(pattern : \type{Pattern}, body : \type{Statement})}
\datadecl{\type{Pattern}}{\ast{PatternWildcard}()
\ALT
\ast{PatternVariable}(variableName : \type{ID})
\ALT
\ast{PatternNumber}(number : \type{NUM})
\ALT
\ast{PatternStructure}(constructorName : \type{ID}, parameters : [\type{ID}])
\ALT
\ast{PatternTuple}(parameters : [\type{ID}])
\ALT
\ast{PatternTimeout}(timeout : \type{NUM})}
\datadecl{\type{Expression}}{\ast{ExprVariable}(variableName : \type{ID})
\ALT
\ast{ExprConstantNumber}(number : \type{NUM})
\ALT
\ast{ExprConstantString}(string : \type{STRING})
\ALT
\ast{ExprChoose}(condition : \type{Expression}, trueExpr : \type{Expression}, falseExpr : \type{Expression})
\ALT
\ast{ExprList}(elements : [\type{Expression}])
\ALT
\ast{ExprRange}(first : \type{Expression}, second : (\type{Expression})?, last : \type{Expression})
\ALT
\ast{ExprTuple}(elements : [\type{Expression}])
\ALT
\ast{ExprStructure}(constructorName : \type{ID}, fieldBindings : [\type{FieldBinding}])
\ALT
\ast{ExprStructureUpdate}(constructorName : \type{ID}, original : \type{Expression}, fieldBindings : [\type{FieldBinding}])
\ALT
\ast{ExprFunctionCall}(functionName : \type{ID}, args : [\type{Expression}])}
\datadecl{\type{ConstructorDeclaration}}{\ast{ConstructorDeclaration}(constructorName : \type{ID}, fieldNames : [\type{ID}])}
\datadecl{\type{FieldBinding}}{\ast{FieldBinding}(fieldName : \type{ID}, value : \type{Expression})}
%
% This file is automatically generated from 03-ast.json using gendoc.py.
% Do not edit by hand.
%


\section{Linter}

La etapa de an\'alisis sem\'antico (o {\em linting}) verifica de manera est\'atica que el programa cumpla las siguientes condiciones.
\bigskip

{\bf Programa principal.}
\begin{itemize}
\item El programa \PUEDE estar completamente vac\'io (sin ninguna definici\'on).
\item Si el programa no est\'a completamente vac\'io, \DEBE tener exactamente una definici\'on de programa
      (ya sea un \ast{DefProgram} o \ast{DefInteractiveProgram}).
\item El programa \NOPUEDE tener dos (o m\'as) definiciones de programa
      (ya sean \ast{DefProgram} o \ast{DefInteractiveProgram}).
\end{itemize}

{\bf Nombres globales.}
\begin{itemize}
\item Hay cinco tipos de nombres globales:
  \begin{itemize}
  \item Nombres de procedimientos, declarados con \texttt{procedure}.
  \item Nombres de funciones, declaradas con \texttt{function}.
  \item Nombres de tipos, declarados con \texttt{type}.
  \item Nombres de constructores.
        Coinciden con el nombre del tipo en la declaraci\'on de un tipo \texttt{record}.
        Pueden coincidir o no con el nombre del tipo en la declaraci\'on de un tipo \texttt{variant},
        en cuyo caso se declaran con \texttt{case}.
  \item Nombres de campos, declarados con \texttt{field}.
  \end{itemize}
\item El programa \NOPUEDE tener dos (o m\'as) funciones con el mismo nombre.
\item El programa \NOPUEDE tener dos (o m\'as) procedimientos con el mismo nombre.
\item El programa \NOPUEDE tener dos (o m\'as) tipos con el mismo nombre.
\item El programa \NOPUEDE tener dos (o m\'as) constructores con el mismo nombre.
\item El programa \NOPUEDE tener dos (o m\'as) campos con el mismo nombre correspondientes a un mismo constructor.
      (P.ej. la definici\'on: \texttt{type A is record \{ field x field x \}} se rechaza).
\item El programa \PUEDE tener dos (o m\'as) campos con el mismo nombre si corresponden a distintos constructores,
      ya sea del mismo tipo o de distintos tipos.
      (P.ej. las definiciones: \texttt{type A is record \{ field x \} type B is record \{ field x \}} se aceptan).
\item El programa \NOPUEDE tener una funci\'on y un campo que tengan el mismo nombre.
\item El programa \PUEDE tener tipos y constructores con el mismo nombre.
      P.ej. la definici\'on: \texttt{type A is variant \{ case A {} \}} se acepta.
      De hecho este es el comportamiento normal cuando se declara un registro:
      \texttt{type A is record \{ \}} define simult\'aneamente un tipo \texttt{A}
      y un constructor \texttt{A}.
\item El programa \PUEDE tener procedimientos con el mismo nombre que un tipo o que un constructor.
\end{itemize}

{\bf Return.}
\begin{itemize}
\item Los procedimientos \NOPUEDEN tener un \texttt{return}.
\item Los programas declarados con \texttt{interactive program} \NOPUEDEN tener un \texttt{return}.
\item Las funciones \DEBEN tener un \texttt{return} como \'ultimo comando del bloque.
\item Los programas declarados con \texttt{program} \PUEDEN tener un \texttt{return} como \'ultimo comando del bloque.
\item \NOPUEDE haber un \texttt{return} en ninguna otra posici\'on salvo
      las mencionadas, es decir,
      como el \'ultimo comando de una funci\'on o como el \'ultimo comando de un \texttt{program}.
\end{itemize}

{\bf Nombres locales.}
\begin{itemize}
\item Hay tres tipos de nombres locales:
  \begin{itemize}
  \item Par\'ametros --- est\'an ligados en la lista de par\'ametros de
        un procedimiento, o en la lista de par\'ametros de una funci\'on, 
        o en la lista de par\'ametros de un constructor al hacer {\em pattern matching}
        con \texttt{switch} (por ejemplo los par\'ametros \texttt{x} e \texttt{y} en
        el comando \texttt{switch (c) \{ Coord(x, y) -> ... \}})
  \item \'Indices --- est\'an ligados por un \texttt{foreach}.
  \item Variables --- su valor se declara en una asignaci\'on \texttt{x := e} o en una asignaci\'on
        a una tupla \texttt{(x1, ..., xN) := e}.
  \end{itemize}
\item Alcance de los nombres locales:
  \begin{itemize}
  \item Par\'ametros de procedimientos y funciones --- locales a todo el cuerpo del procedimiento o funci\'on.
  \item Par\'ametros de constructores en un \texttt{switch} --- locales al bloque a la derecha de la correspondiente flecha \texttt{->}.
  \item \'Indices --- locales al cuerpo del \texttt{foreach}.
  \item Variables --- locales a todo el cuerpo del procedimiento o funci\'on.
  \end{itemize}
\item Los nombres locales \PUEDEN
      coincidir con los nombres globales (nombres de funciones y campos), sin
      que unos opaquen a otros.
\item Los nombres locales \NOPUEDEN coincidir con otros nombres locales que comparten el mismo alcance.
      En particular:
  \begin{itemize}
  \item Las funciones y procedimientos \NOPUEDEN tener nombres de par\'ametros repetidos.
  \item Los \'indices de dos \texttt{foreach} anidados \NOPUEDEN tener el mismo nombre.
  \item Los \'indices de dos \texttt{foreach} que no est\'an anidados \PUEDEN tener el mismo nombre.
  \item Los nombres de variables, \'indices, y par\'ametros \NOPUEDEN coincidir.
  \end{itemize}
\item En una asignaci\'on de tuplas \texttt{let (x$_1$, ..., x$_n$) := e}
      las variables \texttt{x$_1$, ..., x$_n$} \DEBEN ser todas distintas. 
\end{itemize}

{\bf Pattern matching.}
\begin{itemize}
\item Hay cuatro tipos de patrones:
  \begin{itemize}
  \item Patr\'on comod\'in (\texttt{\_}).
  \item Patr\'on constructor (\texttt{C} o alternativamente \texttt{C(x$_1$, ..., x$_n$)}).
  \item Patr\'on tupla (\texttt{(x$_1$, ..., x$_n$)}).
  \item Patr\'on timeout (\texttt{TIMEOUT($n$)}).
  \end{itemize}
\item En un patr\'on constructor, el nombre del constructor en cuesti\'on \DEBE ser un constructor existente de
      alg\'un tipo.
\item En un patr\'on constructor, el constructor \PUEDE tener $0$ par\'ametros, independientemente del n\'umero
      de campos que tenga el constructor correspondiente.
\item Si un patr\'on constructor tiene $1$ o m\'as par\'ametros, el n\'umero \DEBE coincidir con el n\'umero
      de campos del constructor correspondiente.
\item En una secuencia de ramas los patrones constructor \PUEDEN aparecer en cualquier orden, sin respetar necesariamente el orden en que est\'an declarados.
\item Una secuencia de ramas \NOPUEDE tener dos patrones constructor asociados al mismo constructor.
\item Una secuencia de ramas \NOPUEDE tener dos patrones tupla con la misma longitud.
\item Una secuencia de ramas \NOPUEDE tener dos patrones timeout.
\item Una secuencia de ramas \PUEDE tener o no tener un patr\'on comod\'in.
\item Si una secuencia de ramas tiene un patr\'on comod\'in, el patr\'on comod\'in \DEBE estar en la \'ultima rama.
\item Una secuencia de ramas \NOPUEDE tener un patr\'on comod\'in en ninguna otra rama que no sea la \'ultima.
\item Un patr\'on comod\'in \PUEDE ser inalcanzable.
      Es decir, un patr\'on comod\'in puede estar presente incluso si la secuencia de ramas
      cubre todas las alternativas de constructores posibles.
\item En una secuencia de ramas todos los patrones \DEBEN ser compatibles.
      Son patrones compatibles:
      \begin{itemize}
      \item Dos patrones constructor cuyos constructores pertenecen al mismo tipo.
      \item Un patr\'on timeout con cualquier otro constructor del tipo \texttt{\_EVENT}
            (que corresponde a los eventos que pueden darse en un programa interactivo).
      \item El patr\'on comod\'in con cualquier otro patr\'on.
      \end{itemize}
      Dos tipos son incompatibles en cualquier otro caso. En particular, son patrones incompatibles:
      \begin{itemize}
      \item Dos patrones constructor cuyos constructores pertenecen a distintos tipos.
      \item Dos patrones tupla con distinto n\'umero de componentes.
      \item Un patr\'on constructor y un patr\'on tupla.
      \end{itemize}
\item Los patrones de las ramas del \texttt{interactive program}
      \DEBEN ser eventos, es decir:
      patrones timeout,
      patrones constructor del tipo \texttt{\_EVENT},
      o patrones comod\'in.
\item Los de las ramas de los \texttt{switch} \NOPUEDEN ser eventos, es decir:
      pueden ser patrones constructor de tipos que no sean \texttt{\_EVENT},
      patrones tupla,
      o patrones comod\'in.
\end{itemize}

{\bf Expresiones.}
\begin{itemize}
\item Los usos de par\'ametros, \'indices y variables (\ast{ExprVariable}) \PUEDEN no corresponder a par\'ametros, \'indices o variables definidas.
      La restricci\'on de que las variables se pueden usar solamente despu\'es de que se les haya dado un valor es una restricci\'on din\'amica.
\item Cuando se construye una instancia de un tipo (registro o variante) con un constructor (\ast{ExprConstructor}),
      el nombre del constructor \DEBE ser el nombre de un constructor existente.
\item Cuando se construye o se actualiza una instancia con un constructor (\ast{ExprConstructor}, \ast{ExprConstructorUpdate}),
      \NOPUEDE haber nombres de campos repetidos en la lista de {\em bindings}.
\item Cuando se construye o se actualiza una instancia con un constructor (\ast{ExprConstructor}, \ast{ExprConstructorUpdate}),
      los nombres de los campos que aparecen en la lista de {\em bindings}
      \DEBEN ser nombres de campos v\'alidos para dicho constructor. (Correctitud).
\item Cuando se construye una instancia con un constructor (\ast{ExprConstructor}),
      los nombres de los campos que aparecen en la lista de {\em bindings}
      \DEBEN cubrir todos los posibles nombres de campos v\'alidos para dicho constructor. (Completitud).
\item Cuando se construye o se actualiza una instancia con un constructor (\ast{ExprConstructor}, \ast{ExprConstructorUpdate}),
      el constructor \NOPUEDE ser un constructor del tipo \texttt{\_EVENT} (esto t\'ecnicamente depende
      del entorno global en el que se eval\'ue el programa, pero t\'ipicamente
      los eventos son las teclas como \texttt{K\_ENTER}, etc.).
\end{itemize}

{\bf Invocaciones a funciones y procedimientos.}
\begin{itemize}
\item En las invocaciones a procedimientos, el nombre del procedimiento \DEBE corresponder a un procedimiento definido.
\item En las invocaciones a funciones, el nombre de la funci\'on \DEBE corresponder
      o bien a una funci\'on definida o bien al nombre de un campo (usado como funci\'on observadora).
\item El chequeo de coincidencia del n\'umero de argumentos (aridades) se hace est\'aticamente.
      Todos los chequeos de tipos son din\'amicos.
\item El n\'umero de argumentos pasados a un procedimiento \DEBE coincidir con el n\'umero de par\'ametros declarados en su definici\'on.
\item El n\'umero de argumentos pasados a una funci\'on \DEBE coincidir con el n\'umero de par\'ametros declarados en su definici\'on.
\item El n\'umero de argumentos pasados a un campo (usado como funci\'on observadora) \DEBE ser exactamente 1.
\end{itemize}


\end{document}
